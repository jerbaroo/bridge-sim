% Created 2019-06-05 Wed 14:57
% Intended LaTeX compiler: xelatex
\documentclass[11pt]{article}
\usepackage{graphicx}
\usepackage{grffile}
\usepackage{longtable}
\usepackage{wrapfig}
\usepackage{rotating}
\usepackage[normalem]{ulem}
\usepackage{amsmath}
\usepackage{textcomp}
\usepackage{amssymb}
\usepackage{capt-of}
\usepackage{hyperref}
\usepackage{parskip}
\author{Jeremy Barisch-Rooney}
\date{\today}
\title{How sensors can be utilized to build a decision support system for bridge maintenance}
\hypersetup{
 pdfauthor={Jeremy Barisch-Rooney},
 pdftitle={How sensors can be utilized to build a decision support system for bridge maintenance},
 pdfkeywords={},
 pdfsubject={},
 pdfcreator={Emacs 26.2 (Org mode 9.2.3)}, 
 pdflang={English}}
\begin{document}

\maketitle
\tableofcontents


\section{Introduction}
\label{sec:orgb707bef}
The probability of a bridge to fail increases over time until it is no longer
considered safe for use. Maintenance of a bridge is typically carried out
when something goes wrong or according to a preventative maintenance schedule
based on expert knowledge, neither approach making the best use of limited
maintenance resources. Sensors can provide useful real-time information
without the delay or cost of a manual maintenance check. How sensors can be
utilized to build a decision support system for bridge maintenance is the
topic of this thesis.

Sensors on bridges can provide real-time measurements of the responses of the
part of the bridge on which they are installed. Depending on the sensor-type
this measured response may be translation, rotation, vibration or one of many
other types of response. In this thesis the focus is on a single bridge,
bridge 705 in Amsterdam. The reason bridge 705 was chosen is because a 3D
finite element model (FEM) is available for the bridge, and a field test was
conducted where known loads were applied to the bridge and the corresponding
sensor measurements recorded. The FEM is useful so that sensor measurements
for a known load can be simulated without having to conduct a field test, the
measurements from the field test allow us to verify this data generated by
simulation.

A decision support system for bridge maintenance must provide information on
the damage status of the bridge to the user of the system or policy maker.
Thus it is necessary to transform the responses measured by the sensors into
a damage report of the bridge. To accomplish this a model will be built that
generates a damage report based on two methods which will be referred to from
now on as abnormal condition classification (ACC) and similar structure
similar behaviour (SSSB).

The goal of ACC is to determine if the condition of the bridge has deviated
from the normal range of conditions. To build an ACC system it is necessary
to first find out what the range of sensor measurements are during normal
operation of the bridge. This will be achieved by applying a normal range of
loading conditions to the FEM and recording the simulated sensor
measurements. Then a one-class classifier can be applied to the simulated
responses and be used to decide if any subsequent sensor measurements fall
within the expected normal range of responses or not.

The SSSB method is based on the assumption that similar structures should
behave in a similar manner when subjected to the same load. Bridge 705 in
Amsterdam has seven spans each with the same dimensions, ignoring the small
differences due to construction and time in operation. To develop an SSSB
system loads must be "driven" across the bridge in the FEM, then an analysis
must be performed on the difference in sensor measurements from the sensors
at the same position on each substructure.

The end-goal of this thesis is to outline how sensors can be utilized to
build a decision support system for bridge maintenance. To accomplish this a
model will be built (consisting of the AAC and SSSB methods) to estimate from
sensor measurements if damage to the bridge has occured. Research will be
conducted to ascertain what types of sensors and what sensor placement is
optimal to determine a damaged status. A finite element model will be used to
simulate sensor measurements in order to address the lack of available data.
Due to the safety requirements of any bridge it is necessary to quantify the
uncertainty we have in our damage estimates. Once the capabilities and
limitations of the model are understood, we will suggest a decision support
system for policy makers which includes the model and present a cost-benefit
analysis thereof. Finally (stretch-goal) we will investigate how such a
system can be generalized to bridges other than bridge 705.
\section{Background Material}
\label{sec:org5df7368}
\section{Literature Overview}
\label{sec:orgd680adf}
\end{document}
