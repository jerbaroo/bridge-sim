% Created 2019-05-07 Tue 11:39
% Intended LaTeX compiler: xelatex
\documentclass[11pt]{article}
\usepackage{graphicx}
\usepackage{grffile}
\usepackage{longtable}
\usepackage{wrapfig}
\usepackage{rotating}
\usepackage[normalem]{ulem}
\usepackage{amsmath}
\usepackage{textcomp}
\usepackage{amssymb}
\usepackage{capt-of}
\usepackage{hyperref}
\usepackage[table]{xcolor}
\usepackage{parskip}
\usepackage[margin=0.5in, top=0in, bottom=0in]{geometry}
\usepackage{titling}
\author{Jeremy Barisch-Rooney}
\date{}
\title{How can sensors be used in a data-driven decision support system for bridge maintenance?}
\hypersetup{
 pdfauthor={Jeremy Barisch-Rooney},
 pdftitle={How can sensors be used in a data-driven decision support system for bridge maintenance?},
 pdfkeywords={},
 pdfsubject={},
 pdfcreator={Emacs 26.2 (Org mode 9.2.3)}, 
 pdflang={English}}
\begin{document}

\maketitle
\setlength{\droptitle}{-3cm}
\pagenumbering{gobble}

TNO supervisor: Árpád Rózsás - Structural Reliability Researcher, TNO

UvA supervisor: Valeria Krzhizhanovskaya - Researcher Informatics Institute, UvA

The probability of a bridge to fail increases over time until it is no longer
considered safe for use. Maintenance of a bridge is typically carried out
when something goes wrong or according to a preventative maintenance schedule
based on expert knowledge, neither approach making the best use of limited
maintenance resources. Sensor data can provide useful real-time information
without the delay or cost of a manual maintenance check. How sensors can be
used in a data-driven decision support system for bridge maintenance is the
topic of this research. In particular we will focus on bridge 705 in
Amsterdam for which sensor data corresponding to known loads has been
collected, and a 3D finite element model is available. We will build a model
to estimate from sensor data if damage to the bridge has occured and will
investigate what types of sensors and what sensor placement is optimal for
estimating different types of damage. A finite element model will be used to
generate synthetic data in order to address the cold start problem. Due to
the safety requirements of any bridge it is necessary to quantify the
uncertainty we have in our damage estimates. Once the capabilities and
limitations of the model in determining damage are understood, we will
suggest a decision support system for policy makers which includes the model
and present a cost-benefit analysis thereof. Finally (stretch-goal) we will
investigate how such a system can be generalized to bridges other than
bridge 705.

\begin{center}
\begin{tabular}{lrrrl}
Date & WP & Rate & WW & Goal/Note (WP = weeks passed, WW = weeks worked)\\
\hline
Apr 1 & 1 & 0.5 & 0.5 & Reading\\
Apr 8 & 2 & 0.5 & 1.0 & Research question formulated\\
Apr 15 & 3 & 0 & 1.0 & \cellcolor{blue!25} Time off (catching up on work)\\
Apr 22 & 4 & 0.5 & 1.5 & First draft of abstract and timeline\\
Apr 29 & 5 & 0.5 & 2.0 & Investigated bridge damage types and small scale FE model\\
May 6 & 6 & 0.5 & 2.5 & Project presented to TNO staff\\
May 13 & 7 & 0 & 2.5 & \cellcolor{blue!25} Birthday and father visit\\
May 20 & 8 & 0 & 2.5 & \cellcolor{blue!25} Exam\\
May 27 & 9 & 0.5 & 3.0 & Generated synthetic data under normal conditions\\
Jun 3 & 10 & 1 & 4.0 & Completed literature review\\
Jun 10 & 11 & 0.5 & 4.5 & \cellcolor{blue!25} Mother visit\\
Jun 17 & 12 & 1 & 5.5 & Verified synthetic data against sensor measurements\\
Jun 24 & 13 & 1 & 6.5 & Determined sensors containing explanatory information per damage type\\
Jul 14 & 16 & 1 & 9.5 & Determined optimal sensor placement for bridge 705\\
Jul 21 & 17 & 0 & 9.5 & \cellcolor{blue!25} Time off\\
Jul 28 & 18 & 0 & 9.5 & \cellcolor{blue!25} Time off and move closer to the office\\
Aug 19 & 21 & 1 & 12.5 & Calculated damage to bridge 705 with a data-driven model\\
Sep 2 & 23 & 1 & 14.5 & Calculated damage to bridge 705 with a finite element model\\
Sep 9 & 24 & 1 & 15.5 & Determined useful combinations of data-driven and finite element models\\
Sep 30 & 27 & 1 & 18.5 & Quantified measurement and model uncertainty\\
Oct 7 & 28 & 1 & 19.5 & Outlined a decision support system\\
Oct 14 & 29 & 1 & 20.5 & Completed a cost-beneft analysis of the decision support system\\
Oct 21 & 30 & 1 & 21.5 & Started generalizing the model to bridges other than 705\\
Nov 18 & 34 & 1 & 25.5 & Started writing thesis\\
Dec 9 & 37 & 1 & 28.5 & Finished writing draft of thesis\\
Dec 23 & 39 & 0 & 29.5 & \cellcolor{blue!25} Time off\\
Dec 30 & 40 & 0 & 29.5 & \cellcolor{blue!25} Time off\\
Jan 27 & 44 & 1 & 33.5 & Submit\\
\end{tabular}
\end{center}
\end{document}
