% Created 2019-04-04 Thu 11:09
% Intended LaTeX compiler: xelatex
\documentclass[11pt]{article}
\usepackage{graphicx}
\usepackage{grffile}
\usepackage{longtable}
\usepackage{wrapfig}
\usepackage{rotating}
\usepackage[normalem]{ulem}
\usepackage{amsmath}
\usepackage{textcomp}
\usepackage{amssymb}
\usepackage{capt-of}
\usepackage{hyperref}
\author{Jeremy Barisch-Rooney}
\date{\today}
\title{Jeremy's University Courses}
\hypersetup{
 pdfauthor={Jeremy Barisch-Rooney},
 pdftitle={Jeremy's University Courses},
 pdfkeywords={},
 pdfsubject={},
 pdfcreator={Emacs 26.1 (Org mode 9.2.2)}, 
 pdflang={English}}
\begin{document}

\maketitle
\tableofcontents


\section{Department of Computational Science}
\label{sec:orgf4037c9}

\subsection{Introduction to Computational Science (UvA)}
\label{sec:org9a4d46f}

\subsubsection{Objectives}
\label{sec:orgaf48804}

The aim of this course is to provide an overview of Computational Science and
modelling techniques available to the Computational Scientist. We will do this
by considering one specific application (spreading of infectious diseases) and
formulating different types of models and simulations, thus not only exploring
the rich set of modelling paradigms available to us, but also diving deep in the
specific application field. You will also be exposed to recent research results
from the Computational Science Lab, to see other examples of Computational
Science in action.

\subsubsection{Contents}
\label{sec:org49f6142}

Modelling and Simulation, spreading of infectious disease, SIR model and
variants, ODE, PDE, CAs, Networks, ABM. Acquired knowledge will be applied in in
a number of lab assignments, covering the whole chain from model description and
simulation, model verification and validation, to data gathering and analysis .

\subsection{Numerical Algorithms (UvA)}
\label{sec:orga64f017}

\subsubsection{Objectives}
\label{sec:org54c0a92}

To get an overview of basic Numerical Algorithms.

\subsubsection{Contents}
\label{sec:org420183f}

The course gives a broad overview of basic numerical methods for solving Linear
Systems, Least Squares problems, Eigenvalue problems, Nonlinear equations,
Optimization problems, Interpolation and Quadrature problems, and Ordinary
Differential Equations. These methods form the basis for many numerical
algorithms used in computational science and engineering.

\subsection{Agent-based Modelling (UvA)}
\label{sec:orge514ab6}

\subsubsection{Objectives}
\label{sec:org777be03}

At the end of the course, the student will be able to

\begin{itemize}
\item Define what Agent-Based models (ABMs) are.
\item Compare different methods for developing ABMs
\item Properly analyse the output of an ABM
\item Summarize and Compare existing ABM
\item Develop your own Agent-based Model
\item Properly report on your ABM (Scientific paper)
\end{itemize}

\subsubsection{Contents}
\label{sec:org0557935}

The course provides a broad introduction to the field of ABM. It includes a
series of 6 2 hour lectures covering the following topics:

\begin{itemize}
\item Introduction and Classic Models (Epstein, Schelling, Axtell)
\item Agents and AI, a view of agency and agents as defined by the field of
Artificial Intelligence
\item Game Theory \& Agents, covering basic game theory and evolutionary game theory
(Iterated \& Evolutionary Prisoners Dilemma).
\item Discrete Choice Theory for ABM - Logit, Probit Models and more.
\item Sensitivity Analysis Methods for ABM - OFAT, Regression methods and Sobol.
\item Validation for ABM (covering methodologies and challenges in validating ABM).
\end{itemize}

The course also includes practical lectures using (MESA \& Python), these
practical lectures include exercises and questions required each week.

\subsection{Complex System Simulation (UvA)}
\label{sec:org10f6f53}

\subsubsection{Objectives}
\label{sec:org12b9a30}

The student will be able to:

\begin{itemize}
\item Explain the concept of emergence;
\item Name and reason about different types of emergent phenomena, such as chaos,
phase transitions, network connectivity, and complexity;
\item Name and reason about different types of computational models used to study
these phenomena;
\item Implement these models to reproduce a given emergent phenomenon;
\item Compare model outcomes with predictions from (mean-field) theory;
\item Interpret and use the model outcomes in terms of practical applications;
\item Implement and study interventions and what-if scenarios to improve/optimize
with respect to a practical application.
\end{itemize}

\subsubsection{Contents}
\label{sec:orgec04532}

The course has two major components, the first two weeks will include lectures
covering concepts and methods related to Complex System Simulation. The second
two weeks will be group projects (2-3 students), where each group will develop a
simulation of a complex system and conduct research using that simulation.

The group project should reflect the concepts and methods covered in the first
two weeks, however students are also encouraged to self study and suggest
alternative ideas.

Topics to be covered (may change):

\begin{itemize}
\item Introduction to Cellular Automata (1D/2D CA, rule codes, phenomenological
studies, behaviour classes)
\item Self-Organized Criticality (Sandpile Model);
\item Random Networks, Scale Free Networks, Complex Networks;
\item Phase Transitions;
\item (Deterministic) chaos;
\item Complexity;
\item Algorithmic information theory.
\end{itemize}

\subsection{Scientific Computing (UvA)}
\label{sec:org52efab9}

\subsubsection{Objectives}
\label{sec:org1463fde}

\begin{itemize}
\item Knowledge. The student has basic knowledge about solving (examples of) partial
diferential equations and is able to analyze stability and accuracy of a
numerical scheme
\item Skills: The student is able to develop a computational model for solving
examples of partial differential equations (e.g wave equation, diffusion
equations) in (simple) applications
\end{itemize}

\subsubsection{Contents}
\label{sec:org93985fa}

The focus is on developing numerical algorithms (the emphasis is on finite
differencing) to solve prototypical partial differential equations. Examples are
the wave equation, the time-dependent diffusion equation, the Laplace equation
and reaction-diffusion equations. The validation and verification of numerical
schemes for solving these equation will discusssed, Methods for analyzing the
stability and accuracy of the numerical scheme will be discussed.

\section{Artificial Intelligence}
\label{sec:org3afa82d}

\subsection{Evolutionary Computing (VU)}
\label{sec:org0043907}

\subsubsection{Objectives}
\label{sec:org5201ba4}

This course has a threefold objective: 1) To learn about computational methods
based on Darwinian principles of evolution. 2) To illustrate the usage of such
methods as problem solvers and as simulation tools. 3) To gain hands-on
experience in performing computational experiments with evolutionary algorithms.

\subsubsection{Content}
\label{sec:org5fb6e57}

The course is treating various algorithms based on the Darwinian evolution
theory. Driven by natural selection (survival of the fittest), an evolution
process is being emulated and solutions for a given problem are being "bred".
During this course all "dialects" within evolutionary computing are treated
(genetic algorithms, evolution strategies, evolutionary programming, genetic
programming). Applications in optimisation, constraint handling, machine
learning, and robotics are discussed. Specific subjects handled include: various
genetic structures (representations), selection techniques, sexual and asexual
variation operators, (self-)adaptivity. Special attention is paid to
methodological aspects, such as algorithm design and tuning. If time permits,
subjects in Artificial Life will be handled. Hands-on- experience is gained by a
compulsory programming assignment.

\subsection{Computational Intelligence (UvA)}
\label{sec:org259849f}

\subsubsection{Objectives}
\label{sec:orgf251e2c}

The overall aim of this course is to provide knowledge about concepts, theory,
and techniques used in computational intelligence and the know-how to employ
these for making intelligent machines. In particular, to enable students to:
\begin{itemize}
\item gain profound understanding of fundamental computational intelligence
concepts, algorithms, and their implementation;
\item understand the theoretical background of proposed solutions;
\item develop skills in the use of computational intelligence and to demonstrate
this in physical robots or virtual creatures;
\item appreciate relevant current research topics in the theory and practice of
computational intelligence.
\end{itemize}

\subsubsection{Contents}
\label{sec:org2ea9aa5}

Computational intelligence can be positioned as the research area that follows a
bottom-up approach to developing systems that exhibit intelligent behavior in
complex environments. It is often contrasted to the top-down approach followed
by traditional artificial intelligence. Typically, sub-symbolic and
nature-inspired methods are adopted that tolerate incomplete, imprecise and
uncertain knowledge. As a consequence, the resulting approaches allow for
approximate, manageable, robust and resource-efficient solutions.

This course covers nature-inspired techniques such as neural networks,
evolutionary algorithms and swarm intelligence. Special attention is paid to
using such techniques for making autonomous and adaptive machines.

\subsection{Data Mining Techniques (VU)}
\label{sec:orgf877fa8}

\subsubsection{Objectives}
\label{sec:org31972c3}

The aim of the course is that students acquire data mining knowledge and skills
that they can apply in a business environment. How the aims are to be achieved:
Students will acquire knowledge and skills mainly through the following: an
overview of the most common data mining algorithms and techniques (in lectures),
a survey of typical and interesting data mining applications, and practical
assignments to gain "hands on" experience. The application of skills in a
business environment will be simulated through various assignments of the
course.

\subsubsection{Contents}
\label{sec:org78ef43c}

The course will provide a survey of basic data mining techniques and their
applications for solving real life problems. After a general introduction to
Data Mining we will discuss some "classical" algorithms like Naive Bayes,
Decision Trees, Association Rules, etc., and some recently discovered methods
such as boosting, Support Vector Machines, and co-learning. A number of
successful applications of data mining will also be discussed: marketing, fraud
detection, text and Web mining, possibly bioinformatics. In addition to
lectures, there will be an extensive practical part, where students will
experiment with various data mining algorithms and data sets. The grade for the
course will be based on these practical assignments (i.e., there will be no
final examination).

\section{Computer Science}
\label{sec:org0c8bd2b}

\subsection{Performance of Networked Systems (VU)}
\label{sec:orgaef3109}

\subsubsection{Contents}
\label{sec:orgb51a843}

Students will acquire basic knowledge of:
\begin{itemize}
\item performance aspects of networked systems, consisting of servers, services, and
clients
\item performance engineering principles and methods,
\item quantitative models for predicting and optimizing the performance
\end{itemize}
of networked systems,
\begin{itemize}
\item quantitative models for planning capacity of networked systems. Students will
gain experience in engineering and planning performance of networked systems,
and will learn how to tackle practical performance problems arising in the ICT
industry.
\end{itemize}

\subsubsection{Objectives}
\label{sec:org9481032}

Over the past few decades, information and communication technology (ICT) has
become ubiquitous and globally interconnected. As a consequence, our information
and communication systems are expected to process huge amounts of (digital)
information, which puts a tremendous burden on our ICT infrastructure. At the
same time, our modern society has become largely dependent on the
well-functioning of our ICT systems; large-scale system failures and perceivable
Quality of Service (QoS) degradation may completely disrupt our daily lives and
have huge impact on our economy. Motivated by this, the course will focus on
performance-related issues of networked systems. In the first part, we study
capacity planning and modeling for server systems and networks. In the second
part, we study the client side of performance while focusing on web applications
for both desktop and mobile devices. We address questions like:
\begin{itemize}
\item How can we design and engineer networked systems for performance?
\item How can we plan server capacity in networked systems?
\item How can web applications improve performance across wired and wireless
networks?
\end{itemize}

\subsection{Experimental Design \& Data Analysis (VU)}
\label{sec:orga4635ce}

\subsubsection{Objectives}
\label{sec:org9e16856}

In this course the student will get acquainted with the most common experimental
designs and regression models, nonparametric Vrije Universiteit Amsterdam -
Faculteit der Exacte Wetenschappen - M Computer Science (joint degree) -
2017-2018 18-7-2018 - Pagina 39 van 73 tests and bootstrap methods will be
discussed. On completion of this course the student should be able to:
\begin{itemize}
\item design experiments and analyse the results according to the design, - analyse
data using the common ANOVA designs,
\item analyse data using linear regression or a generalized linear regression model,
\item perform basic nonparametric tests,
\item perform bootstrap and permutation tests.
\end{itemize}

\subsubsection{Contents}
\label{sec:org97d9f42}

Regression models try to explain or predict a dependent variable using measured
independent variables. Statistical methods are needed if there is random
variation in the dependent variables. We will discuss multiple linear
regression, analyses of variance (ANOVA), generalized linear regression models.
All methods will be illustrated with practical examples. Especially in the case
of ANOVA it is necessary that the study is well designed in order to draw sound
conclusions from an experiment or survey. In this course a few well known
designs (completely randomized, randomized block etc.) and the associated
analyses of variance are discussed. The remainder of the course is be dedicated
to non-parametric testing methods and bootstrap methods:
\begin{itemize}
\item Wilcoxon test for (one and two samples), - Kolmogorov-Smirnov test (two
samples), - rank correlation tests,
\item permutation and bootstrap tests.
\end{itemize}
All analyses are carried out by using the statistical package R.

\subsection{The Social Web (VU)}
\label{sec:orga40894f}

\subsubsection{Objectives}
\label{sec:org97e6702}

In this course the students will learn theory and methods concerning
communication and interaction in a Web context. The focus is on
distributed user data and devices in the context of the Social Web.
Course content

\subsubsection{Contents}
\label{sec:orgd431034}

This course will cover theory, methods and techniques for:

\begin{itemize}
\item personalization for Web applications;
\item Web user \& context modelling;
\item user-generated content and metadata;
\item multi-device interaction;
\item usage of social-web data.
\end{itemize}

\subsection{Distributed Algorithms (VU)}
\label{sec:org344799d}

\subsubsection{Objectives}
\label{sec:org7c78d20}

The main objective is to provide students with an algorithmic frame of
mind for solving fundamental problems in distributed computing. They
obtain insight into concurrency concepts, and are offered a bird's-eye
view on a wide range of algorithms for basic and important challenges in
distributed systems.

Characteristic of the course is that correctness arguments and
complexity calculations of distributed algorithms are provided in an
intuitive fashion and by means of examples and exercises.

\subsubsection{Contents}
\label{sec:org2dda402}

The following topics are treated in the course: Logical clocks,
snapshots, graph traversal, termination detection, garbage collection,
deadlock detection, routing, election, minimal spanning trees, anonymous
networks, checkpointing, fault tolerance, failure detection, consensus,
mutual exclusion, self-stabilization, blockchains, database transactions

\section{Bioinformatics}
\label{sec:org2828b27}

\subsection{Algorithms in Sequence Analysis (VU)}
\label{sec:org087d7e4}

\subsubsection{Objectives}
\label{sec:org5bbfe3f}

Have you ever wondered how we can track a gene across 3 billion years of
evolution? Sequence alignment can be used to compare genes from humans and
bacteria, using a dynamic programming algorithm. In this course we focus on
algorithms for biological sequences that can be applied to real scientific
problems in biology.

Students will obtain in-depth knowledge about the theory of sequence analysis
methods. They will also develop understanding and skills to apply the algorithms
to protein and DNA sequences. We would like to stress that no biological
knowledge is required to enter this course.

Goals
\begin{itemize}
\item At the end of the course, the student will be aware of the major
\end{itemize}
issues, methodology and available algorithms in sequence analysis.
\begin{itemize}
\item At the end of the course, the student will have hands-on experience in
\end{itemize}
tackling biological problems using sequence analysis algorithms and applying the
general statistical framework of Hidden Markov Models.
\begin{itemize}
\item At the end of the course, the student will be able to implement
\end{itemize}
several of the most important algorithms in sequence analysis. Course content

\subsubsection{Contents}
\label{sec:orgdd08a94}

Theory:
\begin{itemize}
\item Dynamic programming, database searching, pairwise and multiple
\end{itemize}
alignment, probabilistic methods including hidden markov models, pattern
matching, entropy measures, evolutionary models, and phylogeny.

Practical:
\begin{itemize}
\item Programming (in Python) own alignment algorithm based on dynamic
\end{itemize}
programming
\begin{itemize}
\item Reverse translation and dynamic programming
\item Homology searching and pattern recognition using biological and
\end{itemize}
disease examples
\begin{itemize}
\item Multiple alignment of biological sequences
\item Entropy-based functional residues prediction
\item Programming (in Python) own implementation of Hidden Markov Models and
\end{itemize}
using it to predict protein domain structure
\end{document}
