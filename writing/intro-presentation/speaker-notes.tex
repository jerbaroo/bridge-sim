% Created 2019-05-07 Tue 09:12
% Intended LaTeX compiler: xelatex
\documentclass[11pt]{article}
\usepackage{graphicx}
\usepackage{grffile}
\usepackage{longtable}
\usepackage{wrapfig}
\usepackage{rotating}
\usepackage[normalem]{ulem}
\usepackage{amsmath}
\usepackage{textcomp}
\usepackage{amssymb}
\usepackage{capt-of}
\usepackage{hyperref}
\author{Jeremy}
\date{\today}
\title{}
\hypersetup{
 pdfauthor={Jeremy},
 pdftitle={},
 pdfkeywords={},
 pdfsubject={},
 pdfcreator={Emacs 26.2 (Org mode 9.2.3)}, 
 pdflang={English}}
\begin{document}


\section{Slide 1}
\label{sec:org33ac4e2}
Hi, my name is Jeremy Barisch-Rooney. I am from Ireland, half German. I did an
undergraduate degree in software engineering and I work in the same field.

I’m doing a Master degree in Computational Science at the University of
Amsterdam and for my thesis I’m working on a project that is a joint
collaboration between the UoA’s IAS department and TNO.

I will start working on my thesis full time in June, so far I have just been
using some of the days I have available to me on this thesis.

The working title of my research is “how can sensors be used in a data-driven
decision support system for bridges”.
\section{Slide 2}
\label{sec:orgacab840}
My supervisors here are Arpad and Arthur and representing the interests of the
University of Amsterdam is Valeria from the IAS department, who is doing work
in Policy by simulation, which they define as “developing computational models
that help to predict the effect of policy measures, technology development and
new business models in several domains such as energy transition, cities and
mobility, crime and public health.

Some of Valeria’s work related to the monitoring of levees and dikes, part of
this research included an early warning and decision support system based on
sensor data.

For my thesis the research will have some parallels to Valeria’s work, but
instead of levees and dikes I am looking at bridges.
\section{Slide 3}
\label{sec:org14fda84}
Maintenance of assets can be done when the damage occurs, meaning assets
operate at reduced service levels until resources are mobilised to repair the
problem.

Preventative maintenance is an improvement though often inefficient as
maintenance may be done too early or too late.

We hope that sensor data provides enough information to provide a useful
decision support system for bridges to make better use of maintenance
resources.
\section{Slide 4}
\label{sec:org87e978a}
The potential value of sensor-based monitoring includes:

Detecting an incident as early as possible.

Enable condition-based maintenance, providing information on which parts of a
structure are damaged.

Sensors are always collecting more data which can be used to improve the
system.

Together this allows for decision makers to make the best use of limited
maintenance resources via a decision support system. 
\section{Slide 5}
\label{sec:org9bc313c}
Whereas Xuzheng's work is on determining specific hidden parameters of the
structure, this work attempts first to achieve damage detection or an abnormal
state, and then where possible identify which substructure contains the
abnormal state and what type of abnormal state. Naturally the more information
we can make available in the decision support system the better for the
decision maker.
\section{Slide 6}
\label{sec:org96159a6}
Initial focus is on bridge 705 in Amsterdam.

The reason this bridge was chosen is that we have available sensor
measurements from an experiement where bridge 705 was loaded with known loads,
thus we know the loads that correspond to the given sensor measurements.

We also have a 3D FE model which can be used to simulate the condition of the
bridge under different loads, allowing us to create fake or synthetic data.

The sensor measurements can then be used to verify the accuracy of the
synthetic data.
\section{Slide 8}
\label{sec:org4d3b0c1}
I want to classify the bridge state as under normal condition or abnormal.
However we only have sensor measurements corresponding to normal conditions,
and only a very limited set of normal conditions. For obvious reasons we don't
have sensor data corresponding to damage to the bridge.

In cases where only data corresponding to one class is available we can still
do classification. These are called one class classifiers and in the case of
bridge 705 the class is "normal operating conditions".

What I will do is use the available sensor measurements to verify the
synthetic data generated by the FE model. The synthetic data generated by the
FE model will correspond to normal operating conditions, normal loads applied
to the bridge.
\section{Slide 9}
\label{sec:orgdcdda6a}
Once we have the synthetic data representing normal operating conditions we
can project this data to a "cloud" in n-dimensional space. Any testing data
can be projected in a similar manner to see if it falls within the cloud, and
is then classified as a normal operating condition, or outside the cloud, and
classified as an abnormal operating condition.
\section{Slide 10}
\label{sec:orgb1d5310}
Here is an example of a neural cloud in comparison to two other "one class"
classifiers. In contrast to the other two approaches you can see how a sort of
protective barrier or "cloud" surrounds the data points. In the other two
cases you could have a data point which is similar to the data representing
your "normal" class but which is still classified as abnormal.
\section{Slide 11}
\label{sec:org38cb6d3}
Similar substructures should behave the same way when subjected to the same
load. So when a truck is driving over the bridge we would expect the same
response from different sensors.

Bridge 705 has 7 spans. If there is structural damage at one point in the
bridge then we would expect the sensors to behave differently as the truck
passes over each span.

This is a part of the project I'm particularly interested in looking at. This
idea comes from the Sydney Harbour Bridge which has 800 jack arches under
traffic lane 7 and a similar monitoring system was developed, which
succesfully detected three jack arches with issues.

The primary difficulty in the case of bridge 705 is determining the response
of the sensors we would expect as traffic passes over the bridge, for example
consider one truck moving across the bridge. We would expect a similar
response over the next span with a short time delay, this is the relatively
simple case. Now what about two trucks moving in opposite directions, and now
what about heavy traffic?
\section{Slide 12}
\label{sec:orgc634878}
Now I'm going to outline the approach that this thesis will follow.

The approach can be split into four main parts, the generation of synthetic
data, determining optimal sensor placement, creating a data-driven model to
infer damage status of the bridge, and then finally the decision support
support system of which everything above is a part of.
\section{Slide 13}
\label{sec:orgf22c2e2}
To generate synthetic data corresponding to normal operating conditions the
first part is to determine what normal operating conditions are. What is the
distribution of load on the bridge, at a single point in time, and how does
the load change over time? If anyone has knowledge regarding load/traffic
patterns please come find me after the talk, this is valuable data for the
project.

Once the normal operating conditions are determined it is necessary to
generate the synthetic data from a FE model, this means applying the normal
operating loads in a FE model, and measuring the response which we would
expect from different kinds of sensors, translation and strain in particular.

From the experiments performed on bridge 705 then the synthethic data
generated can be verified against sensor measurements.
\section{Slide 14}
\label{sec:orga4d0b45}
Xuzheng has done a lot of work relating to sensor placement. In my case the
definition of optimal will dictate the optimal sensor placement strongly. In
particular the cost of sensors must be included in the evaluation term.

How many sensors of different types should be placed on the structure to
minimise our evaluation function, which will have cost and explanatory
information terms.

This is a general optimisation problem, to constrain the problem space it is
likely desirable to place sensors in the same position on each similar
substructure.
\section{Slide 15}
\label{sec:org2dd9a05}
Determing damage or abnormal conditions is clearly a central part of this
project.

Using the synthetic data which related loads to sensor responses it should be
possible to determine abnormal conditions with some data-driven driven
approach such as the neural cloud or similar substructures techniques.

In Valeria's work related to levees a combination of data-driven and FE model
techniques were used to determine abnormal conditions. Here's a diagram
showing how the data is used both for detection but also for continual
training of the system.
\section{Slide 17}
\label{sec:org27d5ae8}
Finally this entire system would ideally be used as part of a decision support
system so the data can be used to allow decision makers make better use of
limited maintenance resources.

I will outline such a system, similar to what has been presented on the
previous slide. At a high-level such a system consists of three parts, data
acquisition, data analysis, and decision support interface which presents the
results of the analysis.

Of particular interest here is a cost-benefit analysis of such a system,
showing the feasibility of the system to be used in real. Costs include
installation, power, networking, data processing and much more.
\section{Slide 18}
\label{sec:org4a5835c}
It would be nice to show what is necessary to generalise such a system to
other bridges, but there is already a low of work involved so I don't foresee
much if anything being done here.
\section{Slide 19}
\label{sec:org332cb05}
Here is an overview of the deliverables I want to see by the end of the
project, it is really a summary of the previous slides. We have\ldots{}
\section{Slide 20}
\label{sec:orgb6888ae}
\ldots{}
\end{document}
